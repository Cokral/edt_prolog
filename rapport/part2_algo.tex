
\subsection{Algorithme}

\subsubsection{Résolution}

\begin{lstlisting}[language=Prolog, caption=Resolution, captionpos=b]
/**
 * planifier(+Ss, -Cs).
 *
 * @arg S   Listes des seances a planifier
 * @arg C   Listes des creneaux construits
 */
planifier([], []) :- !.
planifier(Ss, [C|Cs]) :-

    member(S, Ss),      % La seance courante
    delete(Ss, S, Ss2), % On l'enleve de la liste

    planifier(Ss2, Cs), % on traite le sous-probleme

    % Creation du creneau et tests ---------------------------------------------

    seance(S, TypeS, _, _),

    % une salle
    salle(L, TailleL),
    accueille(L, TypeL),
    typesCoursIdentiques(TypeS, TypeL), % type de salle valide

    findall(G, groupeSeance(G, S), Gs), % tous les groupes de la seance
    effectifGroupes(Gs, Effectif),
    Effectif =< TailleL, % taille de salle valide

    findall(P, profSeance(P, S), Ps),   % tous les enseignants de la seance

    date(J, M),     % une date
    plage(H, _, _), % une plage horaire

    % test des contraintes (profs, incompatibilite groupes, sequencement)
    % sur cette proposition de creneau
    creneauValide(S, Ps, Gs, H, J, M, L, Cs),

    % Fin creation du creneau et tests -----------------------------------------

    C = [S, H, J, M, L].
\end{lstlisting}

TODO expliquer, parler de sous-problème et récursion

TODO mentionner l'importance de l'ordre des appels aux prédicats pour économiser
lors des retours arrières

\subsubsection{Prédicats utilitaires}

TODO décrire ce qui se passe dans \texttt{creneauValide} et les prédicats
utilisés (mettre leur signature, pas le code)

