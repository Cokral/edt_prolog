
\subsection{Algorithme}

\subsubsection{Résolution}

\begin{lstlisting}[language=Prolog, caption=Resolution, captionpos=b]
/**
 * planifier(+Ss, -Cs).
 *
 * @arg S   Listes des séances à planifier
 * @arg C   Listes des créneaux construits
 */
planifier([], []) :- !.
planifier(Ss, [C|Cs]) :-

    member(S, Ss),      % La séance courante
    delete(Ss, S, Ss2), % On l'enlève de la liste

    planifier(Ss2, Cs), % on traite le sous-problème

    % Création du créneau et tests ---------------------------------------------

    seance(S, TypeS, _, _),

    date(J, M),     % une date
    plage(H, _, _), % une plage horaire

    % une salle
    salle(L, TailleL),
    accueille(L, TypeL),
    typesCoursIdentiques(TypeS, TypeL), % type de salle valide

    findall(G, groupeSeance(G, S), Gs), % tous les groupes de la séance
    effectifGroupes(Gs, Effectif),
    Effectif =< TailleL, % taille de salle valide

    findall(P, profSeance(P, S), Ps),   % tous les enseignants de la séance

    % test des contraintes (profs, incompatibilité groupes, séquencement)
    % sur cette proposition de créneau
    creneauValide(S, Ps, Gs, H, J, M, L, Cs),

    % Fin création du créneau et tests -----------------------------------------

    C = [S, H, J, M, L].
\end{lstlisting}

Voici, dans le code ci-dessus, notre fonction de résolution "planifier(Ss, Cs)". 

Nous avons essayé d'écrire un code qui soit \emph{atomique} au maximum. Chaque fonction effectue une action précise, ou bien est ensuite divisée en plusieurs fonctions qui elles effectueront des requêtes atomiques.

Après avoir récupéré la séance courante, on traite ce sous problème. 
On cherche à diviser le problème pour qu'il traite chaque séance indépendamment. 

Une fois que l'on s'attaque à une seule séance. On va faire les \emph{choix\_nd} décris dans l'algorithme \ref{algo:algo1}.

L'ordre de ces choix est important car la récursivité de Prolog remontera et trouvera de nouvelles valeurs dans l'ordre que nous avons défini.

Tout d'abord, on détermine la date ainsi que la plage horaire via deux \emph{choix\_nd}. 
On commence par ces choix-ci car cela permet de s'assurer que l'on aura les cours sur des créneaux similaires. 
Exemples : le 01/01, de 14h à 15h30, on peut avoir plusieurs séances avec des groupes différents et des salles différentes.

En effet auparavan nous commencions par déterminer la salle et l'emploi du temps était beaucoup plus étendu car les créneaux ne se superposaient que très peu.

Dans un second temps, on va donc trouver une salle qui peut accueillir suffisamment d'élèves ainsi que le bon type de cours. Tant que ces deux variables ne sont pas vraies, Prolog va effectuer un \emph{choix\_nd} de salle.

Ensuite on se penche sur les problèmes de groupes. 
On récupère les groupes associés à la séance, et on vérifie qu'ils ne sont pas trop nombreux par rapport à la salle choisie. 
Si c'est le cas, Prolog va remonter à la fonction précédente, c'est à dire le choix de la salle. 

Il est donc important dans ce cas de les effectuer l'une après l'autre.

Après cela, on récupère le(s) professeur(s) associé(s) à la séance.

Cet ordre permet que, pour un même créneau, on regarde toutes les salles qui correspondent avant de changer de créneau. C'est une bonne façon de régler le problème d'avoir des cours en parallèles.

\subsubsection{Prédicats utilitaires}

\begin{lstlisting}[language=Prolog, caption=creneauValide, captionpos=b]
/**
 * creneauValide(S, Ps, G, H, J, M, L, [Cs]).
 *
 * Définit si un creneau est valide (Pas de conflit avec les créneaux existants)
 *
 * @arg S   La séance
 * @arg Ps  Les enseignants
 * @arg Gs  Les groupes
 * @arg H   La plage horaire
 * @arg J   Le jour
 * @arg M   Le mois
 * @arg L   La salle
 * @arg Cs  Liste de créneaux [S, H, J, M, L]
 */
creneauValide(_, _, _, _, _, _, _, []) :- !.
creneauValide(S, Ps, Gs, H, J, M, L, [C|Cs]) :-
    creneauValideCreneau(S, Ps, Gs, H, J, M, L, C),
    creneauValide(S, Ps, Gs, H, J, M, L, Cs),
    !.
\end{lstlisting}

TODO décrire ce qui se passe dans \texttt{creneauValide} et les prédicats
utilisés (mettre leur signature, pas le code)



