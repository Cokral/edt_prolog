
\section{Solution}

Il convient à présent de définir une première proposition de solution.

Imaginons qu’on dispose d’une fonction $\textsc{Planifier}$ en charge de créer
l’emploi du temps.

Sa signature serait grossièrement :

\begin{description}

\item[Entrées] les séances $S$, les plages horaires $H$, les jours de
travail $J$ et les salles $L$
\item[Sortie] un sous-ensemble de tous les créneaux $C$

\end{description}

\begin{center}

$ \begin{array}{ll}
\textsc{Planifier} : S \times H \times J \times L \quad \not\to \quad C
\end{array}$

\end{center}

On retrouve donc les objets modélisés à la Figure \ref{fig:uml}.

\subsection{Pré-conditions}

Les pré-conditions de la fonction $\textsc{Planifier}$ sont formées par le respect
de la modélisations UML et de l'ajout de détails comme :

\begin{itemize}

    \item les jours sont des nombres entiers appartenant à $\mathbb{N}_{+}$
    \item les plages horaires ne débordent pas des 24h et ne se recouvrent pas

        \[
            \forall(h_0, h_1) \in H, h_0 \leq 0h00, h_1 \leq 23h59
        \]

        \[
            \forall(h_0, h_1),(h_2, h_3) \in H, \;
            h_0 < h_1 \leq h_2 < h_3 \;
            \text{ou} \; h_2 < h_3 \leq h_0 < h_1
        \]

    \item tous les types de cours des séances voient une salle compatible

        \[
            \forall s \in S, \exists l \in L, \text{typeCours}(s) =
            \text{typeCours}(l)
        \]

\end{itemize}

\subsection{Post-conditions}

\begin{itemize}

    \item les créneaux ne débordent pas sur des horaires, des jours ou des
        salles différents des entrés

        \begin{equation}\label{eqn:p1}
            \forall(s, h, j, l) \in \textsc{Planifier}(S, H, L, L),
            s \in S, h \in H, j \in J, l \in L
            \tag{P1}
        \end{equation}

    \item toutes les séances sont affectées à un créneau

        \begin{equation}\label{eqn:p2}
            \forall s \in S,
            \exists(s, h, j, l) \in \textsc{Planifier}(S, H, L, L)
            \tag{P2}
        \end{equation}

    \item toute séance est affectée à une salle qui respecte son type de cours
        et son effectif

        \begin{equation}\label{eqn:p3}
            \forall(s, h, j, l) \in \textsc{Planifier}(S, H, L, L),\;
            \text{typeCours}(s) = \text{typeCours}(l)
            \  \text{et} \  \text{effectif}(s) \leq \text{taille}(l)
            \tag{P3}
        \end{equation}

    \item un enseignant n'a pas deux séances en même temps

        \begin{equation}\label{eqn:p4}
            \forall(s_1, h, j, l_1), (s_2, h, j, l_2)
            \in \textsc{Planifier}(S, H, L, L),\;
            \text{prof}(s_1) \not= \text{prof}(s_2)
            \tag{P4}
        \end{equation}

    \item des groupes incompatibles n'ont pas cours en même temps

        \begin{equation}\label{eqn:p5}
            \begin{split}
                & \forall(s_1, h, j, l_1), (s_2, h, j, l_2)
                \in \textsc{Planifier}(S, H, L, L),\;\\
                & \lnot\text{incompatibles}(g_1, g_2)
                \quad \forall g_1 \in groupes(s_1), \forall g_2 \in groupes(s_2)
            \end{split}
            \tag{P5}
        \end{equation}

    \item les créneaux respectent le séquencement des séances et des matières

        \begin{equation}\label{eqn:p6}
            \begin{split}
                & \forall(s_1, h_1, j_1, l_1), (s_2, h_2, j_2, l_2)
                \in \textsc{Planifier}(S, H, L, L),
                \text{tel que} \  \text{suit}(s_2, s_1)\\
                & \text{alors} \  j_2 \in O
                \  \text{où} \  O = intervalleJours(s_1, s_2)\\
                & \text{et si} \  j_2 = j_1 \  \text{alors} \  h_2 > h_1
            \end{split}
            \tag{P6}
        \end{equation}

\end{itemize}

\subsection{Algorithme non déterministe}

Nous avons choisit de formuler notre solution sous forme d'un algorithme non déterministe.

\begin{algorithm}                    
\caption{basique}      
\label{alg1}                           
\begin{algorithmic}                    % enter the algorithmic environment
    \REQUIRE $C = \emptyset$
    \WHILE{$|S| > 0$}
        \STATE $s \Leftarrow choix_nd(S)$
        \STATE $h \Leftarrow choix_nd(H)$
        \STATE $j \Leftarrow choix_nd(J)$
        \STATE $l \Leftarrow choix_nd(L)$
        \STATE $C \Leftarrow (s, h, j, l)$
        \IF{les conditions sont respectées avec $c$}
            \STATE $C \Leftarrow c \cup C$
            \STATE $S \Leftarrow S / s$
        \ELSE
            \STATE $retourner \perp$
        \ENDIF
    \ENDWHILE
\end{algorithmic}
\end{algorithm}
