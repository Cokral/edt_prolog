\section{Introduction}

Ce rapport présente un premier travail effectué dans le cadre du Mini-Projet
d'Intelligence Artificielle.
Il consiste en la planification de l'emploi du temps à Polytech Nantes.
Ce présent document contient le premier rendu (Résolution générale du problème),
ainsi que la suite du projet (Résolution du problème en Prolog).

Il est également fournit le code source du projet.


\begin{lstlisting}[language=bash, caption=Fichiers, captionpos=b]
README.md
instance.pl
instance_old.pl
main.pl
planifier.pl
run_test.pl
tests\
    instance.pl
    planifier.pl
    utils.pl
utils.pl
\end{lstlisting}

Le programme se lance par la commande suivante :

\begin{lstlisting}[language=bash, caption=Lancer l'algorithme, captionpos=b]
swipl -s main.pl
\end{lstlisting}

Les tests unitaires :

\begin{lstlisting}[language=bash, caption=Lancer les tests, captionpos=b]
swipl -s run_tests.pl
\end{lstlisting}
