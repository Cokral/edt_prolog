
\subsection{\textit{One more thing}}

TODO parler de dynamic

TODO parler des prédicats supplémentaires pour optimiser l'emploi du temps

TODO parler de l'affichage

TODO parler des tests

\subsubsection{Question subsidiaire}

L'algorithme \texttt{planifier(+Ss, +Ds, -Cs)} permet également de planifier des
séances dans un emploi du temps déjà réalisé (\texttt{Ds}).

\begin{lstlisting}[language=Prolog, caption=Exemple d'ajout dans un EDT, captionpos=b,
label={lst:creneauValideCreneau}]

    findall(S, seance(S, _, _, _), Ss), % toutes les seances
    predsort(beforeSeance, Ss, Ss2),
    reverse(Ss2, [X|Ss3]), % X est une seance qu'on ne peut pas planifier tout
                           % de suite
    planifier(Ss3, Cs), % premier emploi du temps

    planifier([X], Cs, Cs2). % ajout de X sur dans l'emploi du temps si possible

\end{lstlisting}

