
\subsection{Ordonnancement et heuristique}

Il est important de remarquer que les séances peuvent être ordonnées avant de
chercher une planification. Les séances d'une même matière se suivant, il est
intéressant de commencer par planifier la première, puis la seconde, etc.

Le prédicat \texttt{indiceSeance(+S, -I)} implémente donc l'heuristique
retournant l'indice d'une séance dans l'arbre des séances séquencées. Cette
heuristique nous permet de trier les séances à priori et de commencer la
la planification des séances de plus petits indices. Elles doivent logiquement
se dérouler en début de période scolaire.

Cet ordonnancement nous permet de rapidement partir dans une branche
relativement bonne. Sans ordonnancement, on peut voir le moteur d'inférence
commencer par placer la dernière séance d'une matière sur le premier
créneau de l'année, ce qui entraine alors de très nombreux retours arrières.

