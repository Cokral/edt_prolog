\subsection{Instance}

Le problème modélisé, il convient maintenant de construire un jeu de données qui
servira à la construction de l'emploi du temps.

La Figure \ref{fig:objet} représente sous forme de diagramme objet l'instance
sur laquelle nous allons travailler. Pour des soucis de lisibilité, le diagramme
est incomplet. Il a été choisi de ne représenter que quelques séances et de ne
pas représenter les dépendances entre elles.

Deux instances sont disponibles :

\begin{description}

    \item[\texttt{instance\_old.pl}] Un petit jeu de données (17 séances) nous
        ayant servit au début du projet et servant actuellement pour les tests
        unitaires

    \item[\texttt{instance.pl}] Représente un semestre complet (TODO : nombre de
        séances, nombre de cours, nombre de profs, etc.)

\end{description}

Pour générer \texttt{instance.pl}, des prédicats ont été utilisés afin de
faciliter la saisie des données en regroupant les séances d'une même matière
dans une même déclaration.

\begin{landscape}

    \begin{figure}[t]
        \includegraphics[keepaspectratio=true,width=26cm]{diagrammeObjet.png}
            \caption{\label{fig:objet} Diagramme objet tronqué}
    \end{figure}

\end{landscape}

