\subsection{UML}

Voici dans la Figure \ref{fig:uml} le diagramme de classe décrivant les données
nécessaires à la gestion de l'emploi du temps.

Comme on peut le constater, on regroupe l'ensemble des données nécessaires et
déterminées à l'avance dans des Séances. A savoir : le(s) groupe(s) d'étudiants
concerné(s), le(s) enseignant(s), la matière, ainsi que le type de cours.
Ces séances vont ensuite être associées à des créneaux par notre solution.
Les créneaux étant le regroupement d'un jour, une plage horaire et une salle.

Un peu noter des détails intéressants, sur Groupe, il y a la notion
d'incompatibilité qui permet de définir lorsqu'il est possible pour deux groupes
d'avoir cours sur une même plage horaire. Sur matière on a la notion de suite,
lorsqu'une matière débute après la fin d'une autre (le Mini-Projet d'IA qui suit
par exemple le cours d'IA).
Et enfin sur séance, on a la notion de suite aussi qui décrit une nombre de
jours minimum et maximum avant le prochain cours de la même matière.

\textbf{Remarque}. Les noms données aux objets de la modélisation sont
légèrement différents de la solution donnée lors de l'indication de
mi-parcours. Nous appelons par exemple "Créneau" l'objet à construire et
regroupant Séance, Salle, Moment.

\begin{landscape}

    \begin{figure}[t]
        \includegraphics[keepaspectratio=true,width=26cm]{diagrammeClasse.png}
            \caption{\label{fig:uml} Modélisation UML}
    \end{figure}

\end{landscape}
